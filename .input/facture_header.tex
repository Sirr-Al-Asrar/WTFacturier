%
%
%   WTFacturier | Header
%  ──────────────────────
%   En tête du fichier tex.
%   Celui contient principalement les extensions et fonctions necessaire à la compilation du document.
%
%
%
% EXTENSIONS :
% ───────────────────
\documentclass[french,11pt]{article}
\usepackage[sfdefault,book]{FiraSans}\renewcommand*\oldstylenums[1]{{\firaoldstyle #1}}
\usepackage[T1]{fontenc}
\usepackage{babel}
\usepackage[utf8]{inputenc}
\usepackage[official]{eurosym}
\usepackage[a4paper]{geometry}
\usepackage{units}
\usepackage{graphicx}
\usepackage{tabu}
\usepackage{fancyhdr}
\usepackage{fp}
\usepackage{scrextend}
\usepackage{xcolor}
%
%
% VARIABLES GENERIQUES :
% ───────────────────
\def\TVA{20}	% Taux de la TVA
\def\TotalHT{0}
\def\TotalTVA{0}
\def\noTVA{%
    \textbf{Total HT} & & & \TotalHT\ \euro \cr {\small TVA non applicable, art 293 B du CGI}
}
%
\makeatletter
%
% FONCTION : addto developpé
% ───────────────────
\newcommand*\eaddto[2]{ %
   \edef\tmp{#2}%
   \expandafter\addto
   \expandafter#1%
   \expandafter{\tmp}%
}
%
% FONCTION : Afficher la liste des produits
% ───────────────────
\newcommand{\AfficheResultat}{%
    \vspace{-0.5em}\\
    \ListeProduits
    \ListeRemises
    \ListeReduction
	\FPadd{\TotalTTC}{\TotalHT}{\TotalTVA}
    \FPeval{\TotalTVA}{\TotalHT * \TVA / 100}
	\FPround{\TotalHT}{\TotalHT}{2}
	\FPround{\TotalTVA}{\TotalTVA}{2}
	\FPround{\TotalTTC}{\TotalTTC}{2}
	\global\let\TotalHT\TotalHT
	\global\let\TotalTVA\TotalTVA
	\global\let\TotalTTC\TotalTTC
	\cr \hline
    \ifthenelse{\equal{\black}{oui}}
    {}
    {\noTVA
        \ifthenelse {\equal{\AfficherTVA}{oui}}
        {Total HT			& & &	\TotalHT	\cr
         TVA \TVA~\% 		& & &	\TotalTVA	\cr}
        {}
    }
}
%
% FONCTION : Ajouter un produit à la liste des produits
% ──────────────────────────────────────────────────────
\def\Titres{}
\def\TitresPremier{0}
\newcommand{\AjouterProduit}[4]{                                % Arguments : Catégorie, Désignation, quantité, prix unitaire HT
                            
   \ifthenelse{\equal{#1}{}}                                        % Si un titre est fournit
       {}                                                               % NON -> Ø
       {\ifthenelse{\equal{#1}{\Titres}}                                % OUI -> Si le titre precedent est le même que le suivant
            {}                                                              % NON -> Ø
            {\ifthenelse{\equal{\TitresPremier}{0}}                         % OUI -> Si c'est le premier Titre de la liste
                {\eaddto\ListeProduits{#1 : & & & \ \ \cr}                      % NON -> On ajoute le titre à la liste
                \def\Titres{#1}                                                 %
                \def\TitresPremier{1}                                           %
                }                                                               %
                {\def\spacev{\vspace{-1ex}}                                     % OUI -> On ajoute le titre à la liste avec un espace réduit
                \eaddto\ListeProduits{\noexpand\spacev\cr#1 : & & & \ \ \cr}    %
                \def\Titres{#1}                                                 %
                }                                                               %
            }                                                               %
       }                                                                %
                                                                    %
    \FPround{\prix}{#4}{2}                                          % Arondir le prix à la deuxième décimal.
	\FPeval{\montant}{#4 * #3}                                      % Multiplier le prix unitaire par la quantité ou le nombre d'heures.
	\FPround{\montant}{\montant}{2}                                 % Arondir le montant global à la deuxieme décimal.
	\FPadd{\TotalHT}{\TotalHT}{\montant}	                        % Ajouter au total de la facture le montant global relevé.
    \eaddto\ListeProduits{                                          % Afficher le produit, sa quantité, son prix unitaire et son montant global.
     \ \ \ \noexpand\small{- #2}	& \prix\ \noexpand\euro &	#3	&	\montant\ \noexpand\euro \ \ \cr %
    }
} 
\newcommand{\ListeProduits}{}
\newcommand{\LigneProduits}{}
%
%
% FONCTION : Ajouter une remise
% ─────────────────────────────
\newcommand{\AjouterRemises}[3]{
	\FPround{\prix}{#3}{2}                                          % Arondir le prix à la deuxième décimal.
	\FPeval{\montant}{#2 * #3}                                      % Multiplier le prix unitaire par la quantité ou le nombre d'heures.
	\FPround{\montant}{\montant}{2}                                 % Arondir le montant global à la deuxieme décimal.
	\FPadd{\TotalHT}{\TotalHT}{-\montant}	                        % Ajouter au total de la facture le montant global relevé.
    \eaddto\ListeRemises{                                           % Afficher le produit, sa quantité, son prix unitaire et son montant global.
     \ \ \ #1	&	- \prix\ \noexpand\euro 	&	#2	&	- \montant\ \noexpand\euro \ \ \cr %
    }
}
\newcommand{\ListeRemises}{}
\newcommand{\LigneRemises}{}
%
%
% FONCTION : Ajouter une reduction
% ────────────────────────────────
\newcommand{\AjouterReduction}[2]{
    %\def\test{\vspace{-1ex}}
	\FPeval{\ristourne}{#2 * \TotalHT / 100}                        % Evaluer le pourcentage donné à partir du total.
    \FPround{\ristourne}{\ristourne}{2}                             % Arondir le montant global à la deuxieme décimal.
    \FPadd{\TotalHT}{\TotalHT}{-\ristourne}	                        % Ajouter au total de la facture la réduction indiquée.
    \eaddto\ListeReduction{%
    \noexpand\spacev\cr Remise :\cr \ \ \ \noexpand\small{- #1}	&	-#2\% &	&	- \ristourne \noexpand\euro \ \ \cr%
    }
}
\newcommand{\ListeReduction}{}
\newcommand{\LigneReduction}{}
%